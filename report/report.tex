\documentclass[conference]{IEEEtran}
% \IEEEoverridecommandlockouts
% The preceding line is only needed to identify funding in the first footnote. If that is unneeded, please comment it out.
\usepackage{cite}
\usepackage{amsmath,amssymb,amsfonts}
\usepackage{algorithmic}
\usepackage{graphicx}
\usepackage{textcomp}
\usepackage{xcolor}
\begin{document}

\title{Using 6LoWPAN for Remote Management of Competitive Autonomous Robots\\
% {\footnotesize \textsuperscript{*}Note: Sub-titles are not captured in Xplore and
% should not be used}
% \thanks{Identify applicable funding agency here. If none, delete this.}
}

\author{\IEEEauthorblockN{Dan Trickey}
\IEEEauthorblockA{\textit{Electronics and Computer Science} \\
\textit{University of Southampton}\\
Southampton, UK \\
dgt1g17@soton.ac.uk}
}

\maketitle

\begin{abstract}
This paper discusses the technologies used and challenges encountered when designing and building a system to remotely start, stop and log autonomous robots in a competition enviroment.
CoAP requests are sent over 6LoWPAN using USB dongles from the competition server to the robots.
\end{abstract}

% \begin{IEEEkeywords}
% component, formatting, style, styling, insert
% \end{IEEEkeywords}

\section{Introduction}

Student Robotics is a UK charity that runs competitive robotics competitions for 16 - 18 year old students in the United Kingdom. Autonomous robots play a pre-defined game in an arena that measures approximately 8 metres by 8 metres. Students' code is loaded onto a USB flash drive and run on a single board computer (SBC). Peripherals such as motors, servos and sensors are connected to and controlled by the SBC over USB.

The competitions are split into a number of matches of a pre-defined length. Robots are started manually using a physical button on them, and execution of code is terminated by a management program on the SBC at the end of the match.

\section{Problems with existing management method}

- early start
- takes ages
- robots crash but we cannot tell

\section{Potential solutions}

- Wifi (nope, IT restrictions, interference)
- IR (nope)
- Wires (only solves one problem)
- Bluetooth Beacons

This paper aims to build a solution based on 6LoWPAN.

\section{Building the dongles}

\subsection{Hardware}

NRF52840 - Why

Dongle - Why

\subsection{Operating System}

Why rRIOT

Port RIOT to dongle

\section{Communication}

\section{Inter-Process Communication on SBC}

DBus

\section{Evaluation of solution}


\bibliographystyle{IEEEtran}
\bibliography{report}

\end{document}
